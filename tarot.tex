
\documentclass[ebook,12pt,twoside,openright,extrafontsizes,final]{memoir}
\usepackage{fontspec}
\usepackage{lettrine}

\setsecnumdepth{chapter}
\setmainfont[  
    Numbers={OldStyle},
    Ligatures={Rare,Historic}
    % ,ItalicFeatures={Colour=990000}
]{EBGaramond}

\medievalpage
\checkandfixthelayout

\begin{document}
\frontmatter
\title{The Game of Triumphs}
\author{Michael Shirk}
\date{MMXIX}
\maketitle

\tableofcontents


Preface
Brief history,
explanation of teh goals of the book.

Read the Prelimiaries. 
Table of num of players x complexity

\mainmatter

\chapter{Preliminaries}
\section{The Deck}
\lettrine{T}{he} Tarot deck is comprised of the four usual suits – but with four, rather
than three Court Cards – and a fifth suit that serves as permanent Trumps.
The first Tarot decks used the same suits that are still used in Italy:
Cups, Coins, Swords, and Batons.\footnote{Current Italian-suited decks include:
the Tarot de Marseille, the Swiss IJJ, and the Tarocco Piemontese.}  In fact,
the Tarot deck was originaly simply the addition of a fifth suit, consisting 
of allegorical subjects, to the ordinary deck of cards.  In the 18th century, 
cardmakers in the German-speaking world began producing Tarot decks with the French suits
of Hearts, Spades, Clubs and Diamonds, and with animals or scenes on the
trump suit rather than the traditional subjects. These French-suited
Tarot decks are now used in France, Germany, Denmark, parts of Switzerland, 
and all the countries of the former Austro-Hungarian Empire.\footnote{These
decks include the Tarot Nouveau, the Industrie und Glück, and the Adler-Cego.}
The Italian suited decks persist only in Italy, parts of Switzerland, and Sicily.

The original deck contained 78 cards: 14 cards each in the ordinary suits, 
21 Trumps, and a special card called ‘The Fool’, or later ‘The Excuse.’
This deck is still used in France and Denmark.  In other areas, however,
the deck is reduced in size by omitting the lower ranking pips in the ordinary
suits. In Austria and countries influenced by it, the Excuse became the highest
Trump, and the deck was reduced to 54 cards: 8 cards per suit (four Court Cards 
and four pips) and the 22 Trumps.

A peculiarity of the deck which goes back even before Tarot to the introduction
of playing cards into Europe, and persists to the present in almost all Tarot
games, is that the pips in the Red or Round suits (Heart and Diamonds, or Cups
and Coins) rank upside down.  Those suits rank, from high to low: King, Queen,
Knight, Jack, 1, 2, 3, 4, 5, 6, 7, 8, 9, 10.

  

[name the courts, indexes, italian and french. Table]

\section{Order of Play}
Unlike most games in the English-speaking world, Tarot games are almost always
played counterclockwise.  The order of play, of course, makes little 
difference; but it is a feature of Italian card games that Tarot has always 
carried with it.

\section{The Deal and Discard}
The cards are usually dealt in groups, rather than singly. This \emph{can} make
a difference; in some games it is the convention (though not the rule) not to
shuffle the deck between deals, but only to cut it. This is supposed to lead
to more interesting hands.

The cards should not be touched until the deal is complete, and then counted.
This way if any mistake was made during dealing, it can be corrected without
having to redeal.

Tarot cards are usually larger than Poker cards, and there are usually more of 
them in hand.  It can be easier to pick the cards up on or two at a time and 
add them to your hand, rather than picking up your entire hand at once and sorting
it in place.  If you're playing with a deck without indices, or with the full 
deck of 78, it might well be impractical to fan your cards out so that you can 
see them all, as you might be used to in other games. Rather, glance through to
get a general idea of your hand, and then fan out only the part that is relevant
to each play. Note also that cards in the Marseille tradition have roman numeral
indexes on the sides, rather than in the corners.

There will be a few cards left over after the deal.  In older games these belong
to the Dealer; in newer ones they will go to the eventual Declarer. That
individual will pick them up and add them to their hand, and then discard 
the same number of cards. The cards in the discard will count for them in the 
end, thus it can sometimes be advantageous to discard point cards.

The usual rules of discarding are that one may \emph{never} discard Honours (Kings, the I and 
XXI of Trumps, and the ’Scuse), and one may only discard other Trumps if necessary to avoid 
discarding an Honour. 

\section{Bidding}
In Tarot games with bidding, the bids rarely alter the "target" of the play, they way 
they do in Spades or Bridge.  Instead, they alter the \emph{conditions} under which one is
willing to aim for a set target—or, in the Austrian family, what the rules of the game
itself will be.  The eventual winner of the bidding will become the Declarer. She alone
(or, perhaps with a partner) will attempt to win the game she has bid and will score
accordingly.  The other players – the Defenders – will attempt to thwart her.

Bidding is done in one round, and proceeds as follows. The Eldest player (the next in
rotation after the dealer) will bid or pass.  If he bids, the next in rotation may make a
higher bid, or pass.  If she makes a higher bid, it goes back to the previous bidder, 
who may \emph{match} the higher bid - or pass. As soon as one of the first two players has
passed, bidding proceeds between the player who did not pass and the third player.  Said 
another way, bidding is always between two players: either the first two who who have not
passed, or between the one who has not passed yet and the next in rotation. 

One you have passed, you may not reenter the bidding.


\section{Tricks}
Tarot games are all Trick-taking games. The most prestigious game of this family
in the English speaking world is Bridge; perhaps the most common in Hearts, due
to its ubiquity on the Windows operating system, is Hearts.

A Trick consists of every player playing one card to the center of the table. 
The first player to play is said to lead to the trick: the suit of the card
they play is called the suit led.  Every other player in rotation must, if
they can, play a card of the same suit from their hand.  If they have no cards
of the suit led, they must play a card from the Trump suit.  If they have no 
remaining trumps either, they may play any card.  The highest Trump, or (if 
no Trumps were played) the highest card of the suit led wins. The player who
played it collects the cards and places them facedown in a pile in front of
him, and then leads to the next trick.

In the older forms of the game, the ’Scuse is the exception to the rules of
following suit.  The holder of the ’Scuse may play it at any point. It never
wins a trick - but it isn’t lost: the player of the ’Scuse, after showing it,
takes it back and adds it to his trick pile, and gives a worthless card in
exchange for it.  If the player never takes a trick, the ’Scuse is forfeited
to the person who one the trick to which it was played.  The ’Scuse is also
forfeited to the winner of the trick if played to the last trick.  If the 
’Scuse is led to a trick, the next player may play anything, and their card
counts as the lead.

In the Austrian games, the ’Scuse is simply the XXII of Trumps.

Tarot games are furthermore \emph{Point Trick} games.  Not only do the Tricks
themselves have value, but certain cards are more valuable in and of themselves.
These cards are the Court Cards, and the Trull: the 1 and 21 of Trumps,
and the Excuse.


\section{Counting Points}
The original method of counting points after the hand, which is still the easiest
to start with, is to count the points for all Court Cards and for the Trull (the I
and XXI of Trumps and the ’Scuse), and then to add one point per trick taken.  In 
fact, in almost all variations the number of cards constituting the ‘trick’ is 
conventionalized, regardless of how many people are playing. Thus, each game will 
specify that the cards are counted ‘in twos,’ ‘in threes,’ or ‘in fours.’

\begin{table}[]
\centering
\begin{tabular}{lc}
\multicolumn{1}{c}{\textbf{Card}} & \textbf{Value} \\
I, XXI, \& ’Scuse & 4 each\\
Kings & 4 each \\
Queens & 3 each\\
Knights & 2 each\\
Jacks & 1 each
\end{tabular}
\end{table}

In other descriptions of Tarot games other counting methods will be described. Although
they may seem rather different, they add up to the same values.

\chapter{“Basic Tarot”}
This game forms the core of all following games. It has been played most 
places where Tarot has permiated, under various names: in Switzerland
it is played as \emph{Troccas}, in Lombardy as \emph{Tarocchi}, and in Piedmont
as \emph{Tarocchi} or \emph{Scarto}, with slight differences.

\medskip
\lettrine{A}{ game} is made up of as many deals as their are players.  After the 
three or four deals of the game, whoever has the lowest score pays a small stake 
to the others, such as buying the next round of drinks.

Played with a 78 card deck. The ’Scuse serves as Excuse.

\section{Three Player}
The Dealer deals 25 cards to each player in batches of 5.  She then takes the
remaining 3 cards into her own hand and discards three.
At the end of the hand, each player counts their cards in groups of 3.  There
are 78 points total in the deck.  Each player's score is recorded as the 
difference between the card points they took and 26.

\section{Four Player}
The players sitting opposite each other are partners, and score as a team.
The Dealer deals 19 cards to each player.  He then takes the remaining 2 into 
his own hand and discards two.
At the end of the hand, each team counts their cards in groups of 4. (The 
Dealer’s discard counts as a full trick.)  There are 72 points total in the 
deck.  Each player's score is recorded as the difference between the card 
points they took and 36.


\chapter{"Classic Tarot" (Ch. 3)}
Gébélin

\chapter{Großtarock (Ch. 4)}
Danish -Perhaps with pot-less scoring as well (see Dummet, 105)

\chapter{Tarock-l’Hombre}
(6.7, 7.1)
Droggn?
(8.28) Chambéry?


\chapter{Le Jeu de Tarot}
c. 1900
Petit en dedans (2000)
Mouches
Score of base + dif (no mult) 1977, Grenoble
base of 10 + diff mult OR 
base of 10, 20, 30, 50 no mult


\chapter{Dreiertarock}

\chapter{Strawman}

\chapter{Königrufen}

\chapter{Strategy}

glossary?

\end{document}